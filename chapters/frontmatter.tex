\maketitle
\makesignature

\ifproject
\begin{abstractTH}
    เป้าหมายหลักของการทําโปรเจคนี้เกิดจาก ในปัจจุบันมีหลักสูตรที่คิดขึ้นและถูกใช้ในมหาวิทยาลัย
    อย่างหลากหลายโดยในแต่ละหลักสูตรนั้นก็จะมีโครงสร้างที่แตกต่างกันออกไป ถึงแม้ว่าจะเป็นหลักสูตรที่ถูก
    ใช้ในคณะหรือสาขาเดียวกันแต่ถ้าเป็นคนละหลักสูตร ก็จะมีโครงสร้างของหลักสูตรที่แตกต่างกันอย่างแน่นอน
    เพราะเกิดการปรับปรุงทั้งเนื้อหาและโครงสร้างของหลักสูตรเพื่อความทันสมัยขององค์ความรู้ ยกตัวอย่างเช่น 
    หลักสูตรปีการศึกษา 2558 และ หลักสูตรปีการศึกษา 2563 ของคณะวิศวกรรมศาสตร์ สาขาวิศวกรรม
    คอมพิวเตอร์ เมื่อนําเอาโครงสร้างของหลักสูตรมาเปรียบเทียบกันดูแล้ว จะพบว่ามีข้อแตกต่างกันในบางส่วน 
    และมีความเหมือนกันในบางส่วนเช่นกัน ซึ่งในแต่ละหลักสูตรก็จะมีความซับซ้อนของโครงสร้างหลักสูตรที่
    แตกต่างกันออกไปตามเกณฑ์ที่แต่ละคณะกําหนด ซึ่งในปัจจุบันเว็บไซต์ของสํานัก-ทะเบียนนั้นมีความล้าสมัย
    ในส่วนที่จะแสดงโครงสร้างของแต่ละหลักสูตร รวมไปถึงบางฟังก์ชันที่เกิดข้อผิดพลาด (bug) และยังใช้
    เวลานานในการประมวลผล เช่น การแสดงข้อมูลหลักสูตรรายบุคคลของนักศึกษาที่ยังไม่มีความละเอียดมากพอ 
    และในการแก้ไขหลักสูตรในแต่ละครั้งของอาจารย์ผู้สอนนั้นมีความยากลําบาก เช่น การเพิ่มวิชาเลือกเข้าไปใน
    หลักสูตรทุกหลักสูตรของสาขาวิศวกรรมคอมพิวเตอร์ จําเป็นที่จะต้องเพิ่มทีละวิชาในทุกหลักสูตรที่ภาควิชามี 
    ทําให้เสียเวลานานพอสมควร ดังนั้นโครงงานนี้จึงมุ่งที่จะแก้ไขปัญหาและอุปสรรคดังกล่าว โดยการสร้าง
    โปรแกรมประยุกต์บนเว็บ (web application) สําหรับจัดการข้อมูลและโครงสร้างในแต่ละหลักสูตรให้มีความ
    เข้าใจง่าย และสะดวกต่อการแก้ไข เพื่อเพิ่มประสิทธิภาพในการใช้งาน และยั่งยืนในอนาคต โดยมีความสามารถ
    ที่จะรองรับหลักสูตรในมหาวิทยาลัย และยืดหยุ่นต่อการใช้เรียกงานต่างๆ รวมไปถึงเพิ่มส่วนที่จะช่วยแสดง
    ข้อมูลการศึกษาของนักศึกษาอย่างละเอียด ให้กับอาจารย์ที่ปรึกษา และตัวนักศึกษาเอง อาทิเช่น การแสดง
    รายวิชาที่ยังไม่ได้ทําการลงทะเบียน และการคํานวณเกรดล่วงหน้า เป็นต้น 
\end{abstractTH}

\begin{abstract}
The abstract would be placed here. It usually does not exceed 350 words
long (not counting the heading), and must not take up more than one (1) page
(even if fewer than 350 words long).

Make sure your abstract sits inside the \texttt{abstract} environment.
\end{abstract}

\iffalse
\begin{dedication}
This document is dedicated to all Chiang Mai University students.

Dedication page is optional.
\end{dedication}
\fi % \iffalse

\begin{acknowledgments}
Your acknowledgments go here. Make sure it sits inside the
\texttt{acknowledgment} environment.

\acksign{2020}{5}{25}
\end{acknowledgments}%
\fi % \ifproject

\contentspage

\ifproject
\figurelistpage

\tablelistpage
\fi % \ifproject

% \abbrlist % this page is optional

% \symlist % this page is optional

% \preface % this section is optional
