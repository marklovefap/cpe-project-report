\chapter{\ifenglish Introduction\else บทนำ\fi}

\section{\ifenglish Project rationale\else ที่มาของโครงงาน\fi}
{จากประสบการณ์ของผู้พัตนาพบว่าหากจะทำการตรวจสอบการสำเร็จการศึกษาของตนเองนั้นกระทำได้ยาก เนื่องจากว่าโครงสร้างของหลักสูตรมีความซับซ้อน ทำความเข้าใจได้อยาก และระบบในปัจจุบันมีความล้าหลังในเรื่องของการตรวจสอบ เเละการเเสดงผลข้อมูลต่างๆที่เกี่ยวกับหลักสูตรการศึกษา ส่งผลให้อาจารย์ เเละนักศึกษาเกิดความลําบากในการใช้งาน รวมถึงการจัดเก็บข้อมูลหลักสูตรที่ยังมีข้อบกพร่อง ยากต่อการค้นหาและนําไปใช้ให้เหมาะสมในปัจจุบัน ซึ่งสังเกตได้จากอาจารย์ผู้รับผิดชอบต้องค้นหาเครื่องมือเพิ่มเติมอาทิชิเช่น Google sheet เพื่อที่จะแก้ปัญหาในในการจัดเก็บข้อมูลหลักสูตรนี้ }


\section{\ifenglish Objectives\else วัตถุประสงค์ของโครงงาน\fi}
%%\begin{enumerate}
  %%  \item
{1 เพื่อสร้าง web application ให้มีความเหมาะสมเเละมีการพัฒนาระบบให้ทันสมัย เพื่อตอบสนองความต้องการของนักศึกษา อาจารย์ที่ปรึกษา และผู้ดูแลหลักสูตรได้อย่างถูกต้อง}
%%\end{enumerate}
%%\begin{enumerate}
{---2 มีระบบการจัดการหลักสูตรที่มีความยืดหยุ่นสูง เพื่อตอบโจทย์ต่อหลักสูตรที่มีโครงสร้างแตกต่างกัน}
%%\end{enumerate}

\section{\ifenglish Project scope\else ขอบเขตของโครงงาน\fi}

\subsection{\ifenglish Hardware scope\else ขอบเขตด้านข้อมูล\fi}
{1 หลักสูตรการศึกษาในระดับปริญญาตรีทุกหลักสูตรของมหาวิทยาลัยเชียงใหม่ ตั้งแต่ปี พ.ศ. 2558 เป็นต้นมา }
\subsection{\ifenglish Software scope\else ขอบเขตด้านซอฟต์แวร์\fi}
{(- นักศึกษา)}
{---1 สามารถแสดงได้ว่านักศึกษาสำเร็จการศึกษาได้อย่างถูกต้อง}
{---2 สามารถแสดงหลักสูตรการศึกษาตามแผนการศึกษา ( แผนการศึกษาแบบปกติ และ แผนการศึกษาแบบสหกิจศึกษา ) ที่นักศึกษาเลือกได้
}
{---3 แสดงผลกระบวนวิชาที่ผ่าน ยังไม่ผ่าน และกำลังลงทะเบียนอยู่ ผ่านทางเว็บไซต์
}
{(- อาจารย์ที่ปรึกษา)}
{---อาจารย์ที่ปรึกษาสามารถตรวจสอบได้ว่านักศึกษาที่ตนดูแลอยู่นั้นสำเร็จการศึกษาแล้วหรือไม่
}
{(- ผู้ดูแลโครงสร้างหลักสูตรการศึกษา)}
{---1 สามารถจัดเก็บหลักสูตรการศึกษาและแสดงผลผ่านทางเว็บไซต์ได้
}
{---2 มีระบบฐานข้อมูลที่สามารถเพิ่ม แก้ไข หรือลบกระบวนวิชาและหลักสูตรการศึกษาได้
}
\section{\ifenglish Expected outcomes\else ประโยชน์ที่ได้รับ\fi}
{(- นักศึกษา)}
{---1 สะดวกในการตรวจสอบการสำเร็จการศึกษาของตนเอง 
}
{---2 ลดความเครียดในการทำความเข้าใจหลักสูตรการศึกษาที่ตนศึกษาอยู่}
{---3 เลือกวางแผนการศึกษาในหลักสูตรการศึกษาตามแบบที่ตนเองสนใจได้}
{---4 มั่นใจได้ว่ากระบวนวิชาที่ตนศึกษาอยู่นั้นเป็นไปตามแผนการศึกษาของหลักสูตรหรือไม่ }
{(- อาจารย์ที่ปรึกษา)}
{---สะดวกในการตรวจสอบความคืบหน้าการสำเร็จการศึกษาของนักศึกษาทั้งแบบรายบุคคล และแบบกลุ่ม }
{(- ผู้ดูแลโครงสร้างหลักสูตรการศึกษา)}
{---1 การเพิ่ม แก้ไข และลบหลักสูตรการศึกษาสามารถทำได้ง่ายขึ้น  และรวดเร็วขึ้น
}
{---2 ลดการใช้กระดาษในการทำเอกสารแสดงผลการสำเร็จการศึกษา 
}
{---3 ไม่จำเป็นต้องใช้เครื่องมือหรือโปรแกรมอื่น ในการจำแนกกระบวนวิชาออกเป็นหมวดหมู่  }
{---4 แสดงหน่วยกิตของกระบวนวิชาในแต่ละหมวดหมู่
}
\section{\ifenglish Technology and tools\else เทคโนโลยีและเครื่องมือที่ใช้\fi}

\subsection{\ifenglish Hardware technology\else เทคโนโลยีด้านฮาร์ดแวร์\fi}
{---1 คอมพิวเตอร์โน้ตบุ๊ครุ่น Lenovo ideapad 320s ที่ใช้ในการเขียนเว็ป
}
{---2 คอมพิวเตอร์โน้ตบุ๊ครุ่น Acer Nitro 5  ที่ใช้ในการเขียนเว็ป
}
\subsection{\ifenglish Software technology\else เทคโนโลยีด้านซอฟต์แวร์\fi}
{---1 Virtual Studio code เป็นเครื่องมือในการเขียนเว็ป}
{---2 HTML เป็นภาษาที่ใช้ในการเขียนเว็ป}
{---3 React และ NodeJS เป็น JavaScript แบบหนึ่งที่ใช้ในการสร้าง web application ทั้ง front-end และ back-end }
{---4 MongDB เป็นเครื่องมือที่ใช้ในการจัดเก็บฐานข้อมูล}
{---5 GraphQL เป็นเครื่องมือที่ใช้ในการสร้าง API }
\section{\ifenglish Project plan\else แผนการดำเนินงาน\fi}


\begin{plan}{6}{2020}{2}{2021}
    \planitem{7}{2020}{8}{2020}{ศึกษาค้นคว้า}
    \planitem{8}{2020}{1}{2021}{ชิล}
    \planitem{2}{2021}{2}{2021}{เผา}
    \planitem{12}{2019}{1}{2022}{ทดสอบ}
\end{plan}

\section{\ifenglish Roles and responsibilities\else บทบาทและความรับผิดชอบ\fi}
{ในส่วนของการศึกษาโครงสร้างหลักสูตรสาขาต่างๆภายในมหาวิทยาลัยเชียงใหม่ตั้งแต่ปีการศึกษา 2558 เป็นต้นมา เพื่อทำความเข้าใจและนําไปใช้ในการออกแบบ data model ให้มีความเหมาะสมต่อการเก็บข้อมูลลงในฐานข้อมูล จึงจำเป็นที่ผู้พัฒนาทั้งสองคนต้องทำงานในส่วนนี้ด้วยกันเพื่อให้เกิดความคิดและความเข้าใจที่ตรงกัน }

{ในส่วน web application ทางผู้พัฒนามีการแบ่งงานออกเป็นสองฝั่ง ได้แก่ }
{---1 ฝั่งหน้าบ้าน (front-end) และฝั่งหลังบ้าน (back-end) ซึ่งในฝั่งหน้าบ้านจําเป็นจะต้องมีความรู้ในเรื่องของ HTML, CSS, JS พอสมควร มีความเข้าใจใน เรื่องของ UX/UI เพื่อการออกแบบโครงสร้างหลักสูตรให้ผู้ใช้สามารถเข้าใจง่าย รวมไปถึงสามารถเขียน requests ส่งไปยังฝั่ง Backend ได้ สามารถจัดการกับ response จากฝั่งหลังบ้านได้ งานในฝั่ง front-end นายอานนท์ รอดตัว จะเป็นผู้รับผิดชอบ}
{---2 ฝั่งหลังบ้าน (back-end) ต้องมีความรู้ในเรื่องของการออกแบบฐานข้อมูล และมีความรู้ในเรื่องการเขียน API สําหรับใช้จัดการกับ requests ที่ทางฝั่งหน้าบ้านส่งมาเพื่อต้องการที่จะนําไปใช้งาน เช่น ต้องเรียกข้อมูลเพื่อนำไปแสดงผล (GET) ต้องการบันทึกลงบนฐานข้อมูล (POST) ต้องการแก้ไขข้อมูลบนฐานข้อมูล (PUT) งานในฝั่ง back-end นายชุติพนธ์ วิมลกาญจนา จะเป็นผู้รับผิดชอบ}
\section{\ifenglish%
Impacts of this project on society, health, safety, legal, and cultural issues
\else%
ผลกระทบด้านสังคม สุขภาพ ความปลอดภัย กฎหมาย และวัฒนธรรม\fi}

{(ด้านสุขภาพจิตที่ดีขึ้น) }

{(-นักศึกษา)}

{จากประสบการณ์ของผู้พัฒนาและการสอบถามจากเพื่อนทั้งในคณะและสาขาเดียวกัน รวมไปถึงต่างคณะแล้วนั้น พบว่าหลายๆคนต่างบอกว่ามีความสับสนในแต่ละหมวดวิชา ทำความเข้าใจได้ยาก จึงทำให้หลายคนไม่ค่อยสนใจและปล่อยผ่านการทำความเข้าใจหลักสูตรของตนเองทิ้งไป  ซึ่งในส่วนนี้จะเกิดผลเสียต่อตัวนักศึกษาเองเนื่องจากจะมีข้อบังคับหรือข้อกำหนดของกระบวนวิชาบางรายวิชาในหมวดต่างที่ต้องลงให้ครบ หรือเลือกลงอย่างใดอย่างหนึ่ง ด้วยข้อกำหนดนี้เองอาจทำให้นักศึกษาลงทะเบียนเรียนผิดพลาดโดยไม่รู้ตัว และส่งผลให้อาจเรียนไม่จบตามปีการศึกษาที่ตนคาดการไว้ เพราะฉะนั้น web application นี้จะช่วยให้นักศึกษามีความเข้าใจในหลักสูตรของตนเอง และเลือกแผนการเรียนที่สนใจได้อย่างถูกต้อง รวมไปถึงเกิดความมั่นใจได้ว่าตนเองจะจบการศึกษาได้อย่างแน่นอนตามที่ web application ได้แสดงผลให้เห็น  }

{(-อาจารย์ที่ปรึกษา)}

{เนื่องจากว่าอาจารย์ที่ปรึกษาบางท่านเป็นอาจารย์ที่ปรึกษาของนักศึกษามากกว่า 1 หลักสูตร ( เป็นอาจารย์ที่ปรึกษาทั้งหลักสูตรปีการศึกษา 2558 และ หลักสูตรปีการศึกษา 2563 ) ซึ่งหลักสูตรแต่ละหลักสูตรจะมีความแต่ต่างกันออกไปมากน้อย จึงอาจทำให้อาจารย์ที่ปรึกษาเกิดความสับสนได้ว่านักศึกษาอยู่คนนั้นอยู่หลักสูตรใด ดังนั้น web application นี้จึงเข้ามาช่วยให้อาจารย์ที่ปรึกษาสามารถตรวจสอบนักศึกษาคนใดๆที่ตนเองดูแลได้ง่ายขึ้นและรวดเร็วยิ่งขึ้น
 }	

{(-ผู้ดูแลโครงสร้างหลักสูตร)}

{จากประสบการณ์ของผู้พัฒนาที่ได้ทำการศึกษาโครงสร้างหลักสูตรของแต่ละคณะ ทุกสาขาวิชา และในทุกปีการศึกษา พบว่าแต่ละหลักสูตรนั้นมีความซับซ้อนของกระบวนวิชาและโครงสร้างหลักสูตรที่แตกต่างกันอย่างมาก และในแต่ละหลักสูตรนั้นจะมีลักษณะเฉพาะของตัวหลักสูตรเองเช่นกัน จึงอาจทำให้ผู้ดูแลนั้นเกิดความเครียดและสับสนเป็นอย่างมากหากต้องทำการแก้ไข เพิ่ม หรือลบกระบวนวิชาภายในหลักสูตร และบางครั้งอาจต้องทำแบบเดิมในหลักสูตรอื่นๆด้วยเป็นจำนวนมาก มิหนำซ้ำการจัดเก็บข้อมูลยังทำได้ยากเนื่องจากว่าต้องเก็บข้อมูลโครงสร้างหลักสูตรลงในกระดาษหรืออาจใช้ดปรแกรมเสริมอย่างเช่น Google sheet ดังนั้นจะดีกว่าไหมหาก web application นี้จะเข้ามาช่วยให้ผู้ดูแลเกิดความสะดวกสบายในการแก้ไข เพิ่ม และลบกระบวนวิชาในหลักสูตรได้ง่ายและกระทำการเพียงครั้งเดียวหากเป็นการทำงานเหมือนเดิมซ้ำๆ รวมถึงช่วยในการจัดเก็บและค้นหาหลักสูตรแต่ละอันได้ง่ายและรวดเร็วกว่ารูปแบบเดิม   
}
		



