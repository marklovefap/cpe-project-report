\chapter{\ifenglish Introduction\else บทนำ\fi}

\section{\ifenglish Project rationale\else ที่มาของโครงงาน\fi}

\CIreply{ยังไม่ได้เกริ่นเรื่องการศึกษาอะไรเลย ก็จะตรวจสอบการสำเร็จการศึกษาแล้ว
ระบุก่อนว่าการศึกษาระดับอุดมศึกษามี requirements ที่เราต้อง fulfill อยู่ ให้เห็นภาพก่อน}
จากประสบการณ์ของผู้พัตนาพบว่าหากจะทำการตรวจสอบการสำเร็จการศึกษาของตนเองนั้นกระทำได้ยาก เนื่องจากว่า\CI{โครงสร้างของหลักสูตรมีความซับซ้อน}{ยังไม่ทันจะเข้าใจอะไร ก็บอกว่าโครงสร้างหลักสูตรมีความซับซ้อนแล้ว} ทำความเข้าใจได้อยาก และระบบในปัจจุบันมีความล้าหลังในเรื่องของการตรวจสอบ เเละการเเสดงผลข้อมูลต่างๆที่เกี่ยวกับหลักสูตรการศึกษา ส่งผลให้อาจารย์ เเละนักศึกษาเกิดความลําบากในการใช้งาน \CI{รวมถึงการจัดเก็บข้อมูลหลักสูตรที่ยังมีข้อบกพร่อง}{รู้ได้อย่างไรว่าบกพร่อง; baseless claims?} ยากต่อการค้นหาและนําไปใช้ให้เหมาะสมในปัจจุบัน ซึ่งสังเกตได้จากอาจารย์ผู้รับผิดชอบต้องค้นหาเครื่องมือเพิ่มเติมอาทิชิเช่น Google sheet เพื่อที่จะแก้ปัญหาในในการจัดเก็บข้อมูลหลักสูตรนี้
\CIreply{ลองไปดู comments ใน abstract แล้วกลับมาคิดใหม่เขียนใหม่}


จากประสบการณ์ของผมที่ได้ประสบพบเจอมา การลงทะเบียนเรียนในแต่ละภาคการศึกษา
ของผมนั้นไม่สามารถลงได้เหมือนตามหลักสูตรที่มีมาให้เนื่องจากมีบางวิชาที่ไม่ผ่าน และบางวิชาก็
\CIreply{เขียนเป็น third-person}
\CI{อาจไม่ได้เปิดในเทอมนั้น}{ระบบของเราจะแก้ปัญหาตรงนี้หรือไม่} ทำให้ต้องปรับแผนการเรียนใหม่ อีกทั้งการลงเรียนวิชาไม่ตรงตามหลักสูตร
ยังทำให้เกิดความสับสนในวิชาตัวต่อไปเป็นอย่างมาก ผมจึงได้คิดหาวิธีการแก้ปัญหาที่สามารถช่วยลด
ความสับสนในการวางแผนการเรียนได้ ก็คือการทำเว็ปแอปพลิเคชั่นที่สามารถช่วย\CI{จัดการวางแผนการ
เรียน}{จริงหรือเปล่า}ตามความต้องการของผู้ใช้งานได้ โดยพื้นฐานของแผนมาจากหลักสูตรของผู้ใช้งาน การทำใน
รูปแบบเว็ปแอปพลิเคชั่นนั้นสามารถใช้งานได้ทุกแพลตฟอร์ม สะดวกต่อการใช้งาน เว็ปจะสามารถ
เลือกหลักสูตรแล้วแสดงผลออกมาในรูปแบบของผังงาน (flowchart) จากนั้นผู้ใช้งานก็จะสามารถ
\CIreply{ปรับเปลี่ยนแก้ไขได้}{ปรับเปลี่ยนแก้ไขอะไร} เพื่อที่นำผังหลังจากแก้ไขมาใช้วิเคราะห์การลงทะเบียนเรียนต่อไปได้
\CIreply{สองย่อหน้านี้ เป็นประเด็นที่พูดถึงปัญหาเหมือนกัน แต่พูดสองรอบ
เอามาเรียงประเด็นรวมกันให้สละสลวยได้ไหม}

\section{\ifenglish Objectives\else วัตถุประสงค์ของโครงงาน\fi}
\begin{enumerate}
    \item เพื่อสร้าง web application \CI{ให้มีความเหมาะสมเเละมีการพัฒนาระบบให้ทันสมัย}{จะทำอะไร} เพื่อตอบสนองความต้องการของนักศึกษา อาจารย์ที่ปรึกษา และผู้ดูแลหลักสูตรได้อย่างถูกต้อง
    \item มีระบบการจัดการหลักสูตรที่มีความ\CI{ยืดหยุ่น}{ยืดหยุ่นอย่างไร}สูง เพื่อตอบโจทย์ต่อหลักสูตรที่มีโครงสร้างแตกต่างกัน
\end{enumerate}
\CIreply{พอเราอธิบายที่มาของปัญหาไม่ชัดเจน วัตถุประสงค์ก็ไม่ชัดเจนตามไปด้วย}

\section{\ifenglish Project scope\else ขอบเขตของโครงงาน\fi}

\subsection{\ifenglish Information scope\else ขอบเขตด้านข้อมูล\fi}
\begin{enumerate}
    \item หลักสูตรการศึกษาในระดับปริญญาตรีทุกหลักสูตรของมหาวิทยาลัยเชียงใหม่ ตั้งแต่ปี พ.ศ. 2558 เป็นต้นมา 
\end{enumerate}

%%\subsection{\ifenglish Hardware scope\else ขอบเขตด้านฮาร์ดแวร์\fi}

\subsection{\ifenglish Software scope\else ขอบเขตด้านซอฟต์แวร์\fi}

\begin{itemize}
\item นักศึกษา 

\begin{enumerate}
    \item สามารถแสดงได้ว่านักศึกษาสำเร็จการศึกษาได้อย่างถูกต้อง
    \item สามารถแสดงหลักสูตรการศึกษาตามแผนการศึกษา (เช่น แผนการศึกษาแบบปกติ และ แผนการศึกษาแบบสหกิจศึกษา) ที่นักศึกษาเลือกได้
    \item สามารถแสดงหลักสูตรการน ยังไม่ผ่าน และกำลังลงทะเบียนอยู่ \CI{ผ่านทางเว็บไซต์}{แล้ว items ก่อนหน้านี้ แสดงผ่านทางไหน}
\end{enumerate}

\item อาจารย์ที่ปรึกษา 

\begin{enumerate}
    \item อาจารย์ที่ปรึกษาสามารถตรวจสอบได้ว่านักศึกษาที่ตนดูแลอยู่นั้นสำเร็จการศึกษาแล้วหรือไม่
\end{enumerate}

\item ผู้ดูแลโครงสร้างหลักสูตรการศึกษา

\begin{enumerate}
    \item สามารถจัดเก็บหลักสูตรการศึกษาและแสดงผลผ่านทางเว็บไซต์ได้
    \item \CI{มีระบบฐานข้อมูล}{เป็นขอบเขตอย่างไร}ที่สามารถเพิ่ม แก้ไข หรือลบกระบวนวิชาและหลักสูตรการศึกษาได้
\end{enumerate}
\end{itemize}
\CIreply{ควรเพิ่มว่าสิ่งใดไม่อยู่ใน scope เพื่อให้ชัดเจนขึ้น}

\section{\ifenglish Expected outcomes\else ประโยชน์ที่ได้รับ\fi}
\CIreply{ฝากใส่ itemize ด้วยครับ}
-นักศึกษา 

\begin{enumerate}
    \item สะดวกในการตรวจสอบการสำเร็จการศึกษาของตนเอง 
    \item ลดความเครียดในการทำความเข้าใจหลักสูตรการศึกษาที่ตนศึกษาอยู่ 
    \item เลือกวางแผนการศึกษาในหลักสูตรการศึกษาตามแบบที่ตนเองสนใจได้
    \item มั่นใจได้ว่ากระบวนวิชาที่ตนศึกษาอยู่นั้นเป็นไปตามแผนการศึกษาของหลักสูตรหรือไม่ 
    
\end{enumerate}

-อาจารย์ที่ปรึกษา 

\begin{enumerate}
    \item สะดวกในการตรวจสอบความคืบหน้าการสำเร็จการศึกษาของนักศึกษาทั้งแบบรายบุคคล และแบบกลุ่ม 
\end{enumerate}

-ผู้ดูแลโครงสร้างหลักสูตรการศึกษา

\begin{enumerate}
    \item การเพิ่ม แก้ไข และลบหลักสูตรการศึกษาสามารถทำได้ง่ายขึ้น  และรวดเร็วขึ้น
    \item ลดการใช้กระดาษในการทำเอกสารแสดงผลการสำเร็จการศึกษา 
    \item ไม่จำเป็นต้องใช้เครื่องมือหรือโปรแกรมอื่น ในการจำแนกกระบวนวิชาออกเป็นหมวดหมู่ 
    \item แสดงหน่วยกิตของกระบวนวิชาในแต่ละหมวดหมู่
    
\end{enumerate}

\section{\ifenglish Technology and tools\else เทคโนโลยีและเครื่องมือที่ใช้\fi}

\subsection{\ifenglish Hardware technology\else เทคโนโลยีด้านฮาร์ดแวร์\fi}

\begin{enumerate}
    \item คอมพิวเตอร์โน้ตบุ๊ครุ่น Lenovo ideapad 320s ที่ใช้ในการเขียนเว็บ
    \item คอมพิวเตอร์โน้ตบุ๊ครุ่น Acer Nitro 5  ที่ใช้ในการเขียนเว็บ
\end{enumerate}

\subsection{\ifenglish Software technology\else เทคโนโลยีด้านซอฟต์แวร์\fi}

\begin{enumerate}
    \item \CI{Virtual}{spelling} Studio code เป็นเครื่องมือในการเขียน\CI{เว็ป}{sp.}
    \item HTML เป็นภาษาที่ใช้ในการเขียนเว็ป
    \item React และ NodeJS เป็น JavaScript แบบหนึ่งที่ใช้ในการสร้าง web application ทั้ง front-end และ back-end  
    \item MongDB เป็นเครื่องมือที่ใช้ในการจัดเก็บฐานข้อมูล
    \item GraphQL เป็นเครื่องมือที่ใช้ในการสร้าง API 
\end{enumerate}
\CIreply{ใส่ reference ถึงเครื่องมือต่างๆ ด้วย}

\section{\ifenglish Project plan\else แผนการดำเนินงาน\fi}

\CIreply{missing}
%%\begin{plan}{6}{2020}{2}{2021}
  %%  \planitem{7}{2020}{8}{2020}{ศึกษาค้นคว้า}
   %% \planitem{8}{2020}{1}{2021}{ชิล}
   %% \planitem{2}{2021}{2}{2021}{เผา}
   %% \planitem{12}{2019}{1}{2022}{ทดสอบ}
%%\end{plan}

\section{\ifenglish Roles and responsibilities\else บทบาทและความรับผิดชอบ\fi}

ในส่วนของการศึกษาโครงสร้างหลักสูตรสาขาต่างๆภายในมหาวิทยาลัยเชียงใหม่ตั้งแต่ปีการศึกษา 2558 เป็นต้นมา เพื่อทำความเข้าใจและนําไปใช้ในการออกแบบ data model ให้มีความเหมาะสมต่อการเก็บข้อมูลลงในฐานข้อมูล จึงจำเป็นที่ผู้พัฒนาทั้งสองคนต้องทำงานในส่วนนี้ด้วยกันเพื่อให้เกิดความคิดและความเข้าใจที่ตรงกัน 
	
ในส่วน web application ทางผู้พัฒนามีการแบ่งงานออกเป็นสองฝั่ง ได้แก่ 

\begin{enumerate}
    
    \item ฝั่งหน้าบ้าน (front-end) \CI{และฝั่งหลังบ้าน (back-end)}{?} ซึ่งในฝั่งหน้าบ้านจําเป็นจะต้องมีความรู้ในเรื่องของ HTML, CSS, JS พอสมควร มีความเข้าใจใน เรื่องของ UX/UI เพื่อการออกแบบโครงสร้างหลักสูตรให้ผู้ใช้สามารถเข้าใจง่าย รวมไปถึงสามารถเขียน requests ส่งไปยังฝั่ง Backend ได้ สามารถจัดการกับ response จากฝั่งหลังบ้านได้ งานในฝั่ง front-end นายอานนท์ รอดตัว จะเป็นผู้รับผิดชอบ
    \item ฝั่งหลังบ้าน (back-end) ต้องมีความรู้ในเรื่องของการออกแบบฐานข้อมูล และมีความรู้ในเรื่องการเขียน API สําหรับใช้จัดการกับ requests ที่ทางฝั่งหน้าบ้านส่งมาเพื่อต้องการที่จะนําไปใช้งาน เช่น ต้องเรียกข้อมูลเพื่อนำไปแสดงผล (GET) ต้องการบันทึกลงบนฐานข้อมูล (POST) ต้องการแก้ไขข้อมูลบนฐานข้อมูล (PUT) งานในฝั่ง back-end นายชุติพนธ์ วิมลกาญจนา จะเป็นผู้รับผิดชอบ

\end{enumerate}


\section{\ifenglish%
Impacts of this project on society, health, safety, legal, and cultural issues
\else%
ผลกระทบด้านสังคม สุขภาพ ความปลอดภัย กฎหมาย และวัฒนธรรม\fi}


ด้านสุขภาพจิตที่ดีขึ้น 

\begin{enumerate}
    \item นักศึกษา 
	
    จากประสบการณ์ของผู้พัฒนาและ\CI{การสอบถามจากเพื่อนทั้งในคณะและสาขาเดียวกัน รวมไปถึงต่างคณะแล้ว}{ควรนำผลการสอบถามไปเขียนโดยละเอียดในบทหลังๆ}นั้น พบว่าหลายๆคนต่างบอกว่ามีความสับสนในแต่ละหมวดวิชา ทำความเข้าใจได้ยาก จึงทำให้หลายคนไม่ค่อยสนใจและปล่อยผ่านการทำความเข้าใจหลักสูตรของตนเองทิ้งไป  ซึ่งในส่วนนี้จะเกิดผลเสียต่อตัวนักศึกษาเองเนื่องจากจะมีข้อบังคับหรือข้อกำหนดของกระบวนวิชาบางรายวิชาในหมวดต่างที่ต้องลงให้ครบ หรือเลือกลงอย่างใดอย่างหนึ่ง ด้วยข้อกำหนดนี้เองอาจทำให้นักศึกษาลงทะเบียนเรียนผิดพลาดโดยไม่รู้ตัว และส่งผลให้อาจเรียนไม่จบตามปีการศึกษาที่ตนคาดการไว้ เพราะฉะนั้น web application นี้จะช่วยให้นักศึกษามีความเข้าใจในหลักสูตรของตนเอง และเลือกแผนการเรียนที่สนใจได้อย่างถูกต้อง รวมไปถึงเกิดความมั่นใจได้ว่าตนเองจะจบการศึกษาได้อย่างแน่นอนตามที่ web application ได้แสดงผลให้เห็น  

    
    \item อาจารย์ที่ปรึกษา 
	
    เนื่องจากว่าอาจารย์ที่ปรึกษาบางท่านเป็นอาจารย์ที่ปรึกษาของนักศึกษามากกว่า 1 หลักสูตร ( เป็นอาจารย์ที่ปรึกษาทั้งหลักสูตรปีการศึกษา 2558 และ หลักสูตรปีการศึกษา 2563 ) ซึ่งหลักสูตรแต่ละหลักสูตรจะมีความแต่ต่างกันออกไปมากน้อย จึงอาจทำให้อาจารย์ที่ปรึกษาเกิดความสับสนได้ว่านักศึกษาอยู่คนนั้นอยู่หลักสูตรใด ดังนั้น web application นี้จึงเข้ามาช่วยให้อาจารย์ที่ปรึกษาสามารถตรวจสอบนักศึกษาคนใดๆที่ตนเองดูแลได้ง่ายขึ้นและรวดเร็วยิ่งขึ้น
    \CIreply{เกี่ยวกับสุขภาพจิตอย่างไร}

   
    \item ผู้ดูแลโครงสร้างหลักสูตร 
	
    จากประสบการณ์ของผู้พัฒนาที่ได้ทำการศึกษาโครงสร้างหลักสูตรของแต่ละคณะ ทุกสาขาวิชา และในทุกปีการศึกษา พบว่าแต่ละหลักสูตรนั้นมีความซับซ้อนของกระบวนวิชาและโครงสร้างหลักสูตรที่แตกต่างกันอย่างมาก และในแต่ละหลักสูตรนั้นจะมีลักษณะเฉพาะของตัวหลักสูตรเองเช่นกัน จึงอาจทำให้ผู้ดูแลนั้นเกิดความเครียดและสับสนเป็นอย่างมากหากต้องทำการแก้ไข เพิ่ม หรือลบกระบวนวิชาภายในหลักสูตร และบางครั้งอาจต้องทำแบบเดิมในหลักสูตรอื่นๆด้วยเป็นจำนวนมาก มิหนำซ้ำการจัดเก็บข้อมูลยังทำได้ยากเนื่องจากว่าต้องเก็บข้อมูลโครงสร้างหลักสูตรลงในกระดาษหรืออาจใช้ดปรแกรมเสริมอย่างเช่น Google sheet ดังนั้น\CI{จะดีกว่าไหม}{ภาษาพูด}หาก web application นี้จะเข้ามาช่วยให้ผู้ดูแลเกิดความสะดวกสบายในการแก้ไข เพิ่ม และลบกระบวนวิชาในหลักสูตรได้ง่ายและกระทำการเพียงครั้งเดียวหากเป็นการทำงานเหมือนเดิมซ้ำๆ รวมถึงช่วยในการจัดเก็บและค้นหาหลักสูตรแต่ละอันได้ง่ายและรวดเร็วกว่ารูปแบบเดิม   
\end{enumerate}
