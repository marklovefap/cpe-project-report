\maketitle
\makesignature

\ifproject
\begin{abstractTH}
    เป้าหมายหลักของการทําโปรเจคนี้เกิดจาก ในปัจจุบันมีหลักสูตรที่คิดขึ้นและถูกใช้ในมหาวิทยาลัยอย่าง หลากหลายโดยใน
    แต่ละหลักสูตรนั้นก็จะมีโครงสร้างที่แตกต่างกันออกไป ถึงแม้ว่าจะเป็นหลักสูตรที่ถูกใช้ในคณะ หรือสาขาเดียวกันแต่ถ้าเป็นคนละ
    หลักสูตร ก็จะมีโครงสร้างของหลักสูตรที่แตกต่างกันอย่างแน่นอนเพราะเกิดการปรับปรุงทั้งเนื้อหาเเละโครงสร้างของหลักสูตรเพื่อ
    ความทันสมัยขององค์ความรู้ ยกตัวอย่างเช่น หลักสูตรปี การศึกษา 2558 และ หลักสูตรปี การศึกษา 2563 ของคณะ
    วิศวกรรมศาสตร์ สาขาวิศวกรรมคอมพิวเตอร์ เมื่อนําเอาโครงสร้างของหลักสูตรมาเปรียบเทียบกันดูแล้ว จะพบว่ามีข้อแตกต่างกันใน
    บางส่วน และมีความเหมือนกันในบางส่วนเช่นกัน ซึ่งในแต่ละหลักสูตรก็จะมีความซับซ้อนของโครงสร้างหลักสูตรที่แตกต่างกันออกไป
    ตามเกณฑ์ที่แต่ละคณะกําหนด ซึ่งในปัจจุบันเว็บไซต์ของสํานักทะเบียนนั้นมีความล้าสมัยในส่วนที่จะแสดงโครงสร้างของแต่ละ
    หลักสูตร รวมไปถึงบางฟังก์ชันที่เกิดข้อผิดพลาด (bug) และยังใช้เวลานานในการประมวลผล เช่น การแสดงข้อมูล หลักสูตร
    รายบุคคลของนักศึกษาที่ยังไม่มีความละเอียดมากพอ และในการแก้ไขหลักสูตรในแต่ละครั้งของอาจารย์ ผู้สอนนั้นมีความยากลําบาก
    เช่น การเพิ่มวิชาเลือกเข้าไปในหลักสูตรทุกหลักสูตรของสาขาวิศวกรรมคอมพิวเตอร์ จําเป็นที่จะต้องเพิ่มทีละวิชาในทุกหลักสูตรที่
    ภาควิชามีทําให้เสียเวลานานพอสมควร ดังนั้นโครงงานนี้จึงมุ่งที่จะ แก้ไขปัญหาและอุปสรรคดังกล่าว โดยการสร้างโปรแกรมประยุกต์
    บนเว็บ (web application) สําหรับจัดการข้อมูลและโครงสร้างในแต่ละหลักสูตรให้มีความเข้าใจง่ายและสะดวกต่อการแก้ไข เพื่อ
    เพิ่มประสิทธิภาพในการใช้ งาน โดยมีความสามารถที่จะรองรับหลักสูตรในมหาวิทยาลัย รวมไปถึงเพิ่มส่วนที่จะช่วยเเสดงข้อมูล
    การศึกษาของ นักศึกษาอย่างละเอียดให้กับอาจารย์ที่ปรึกษาและตัวนักศึกษาเอง อาทิเช่น การแสดงรายวิชาที่ยังไม่ได้ทําการ
    ลงทะเบียน และการคํานวณเกรดล่วงหน้า เป็นต้น
\end{abstractTH}

\begin{abstract}
    The main goal of doing this project comes from At present, 
    there are courses that are invented and used in universities like There are many different courses 
    in which each course is structured differently. Although it is a course that is used in the Faculty or 
    the same branch, but if it is a different course There will definitely be a different course structure 
    because both the content and the course structure are updated for the modernization of the body of knowledge. 
    For example, the 2015 academic year program and the 2020 academic year program of the Faculty of Engineering. 
    Computer Engineering When comparing the structure of the curriculum You will find that there are some differences. 
    And they are the same in some parts as well. Each course has a different complexity of course structure according to 
    the criteria set by each faculty. Currently, the registrar's website is outdated in terms of showing the structure of 
    each course, including some functions that have bugs (bugs) and take a long time to process, such as displaying information.
     Individual courses of students that are not yet detailed enough and in the course of each revision of the teacher Instructors 
     have difficulties such as adding elective courses to all courses in computer engineering. It is necessary to add one 
     subject at a time to every course the department has, which takes a considerable amount of time. Therefore, 
     this project aims to solve such problems and obstacles By creating a web application (web application) to manage 
     the data and structure in each course to be easy to understand and easy to edit. To increase the efficiency of 
     use with the ability to support the courses in the university including adding a section that will help display 
     the educational information of students in detail to the advisors and students themselves, such as showing courses 
     that have not yet been registered and calculating grades in advance, etc.
\end{abstract}

\iffalse
\begin{dedication}
Dedication page is optional.
\end{dedication}
\fi % \iffalse

\begin{acknowledgments}
    โครงงานนี้ได้รับความกรุณาจาก อ.ดร.ชินวัตร อิศราดิสัยกุล อาจารย์ที่ปรึกษาที่ได้สละเวลาให้ความช่วยเหลือ ให้คำแนะนำ ให้ความรู้และแนวคิดต่างๆ รวมไปถึงขอขอบคุณอาจารย์คณะกรรมการทั้ง 
    อ.ดร.พฤษภ์ บุญมา และ ผศ.ดร.นวดนย์ คุณเลิศกิจ ที่ให้คำแนะนำต่างๆ รวมไปถึงเพื่อนๆที่ให้กำลังใจและคำแนะนำที่ดีตลอดการทำโครงงานที่ผ่านมา จนทำโครงงานเล่มนี้ออกมาได้อย่างเสร็จสมบูรณ์
    นอกจากนี้ผู้จัดทําขอขอบพระคุณบิดา-มารดาที่ได้ให้ชีวิต เลี้ยงดูสั่งสอน และส่งเสียให้ผู้จัดทําได้ศึกษาเล่าเรียนจนจบหลักสูตรปริญญาตรี หลักสูตรวิศวกรรมศาสตร์บัณฑิต ซึ่งท่านได้ให้กําลังใจในวันที่ยากลําบากและยังเป็นแรงผลักดันให้สร้างสรรค์ผลงานและมุ่งมั่นจนทําให้โครงงานนี้สําเร็จ รวมทั้งขอขอบพระคุณอีกหลายๆท่านที่ไม่ได้เอ่ยนามมา ณ ที่นี้ ที่ได้ให้ความช่วยเหลือตลอดมา หากหนังสือโครงงานเล่มนี้ผิดพลาดประการใด ทางผู้จัดทําขอยอมรับด้วยความยินดี

\acksign{2022}{10}{10}
\end{acknowledgments}%
\fi % \ifproject

\contentspage

\ifproject
\figurelistpage

\tablelistpage
\fi % \ifproject

% \abbrlist % this page is optional

% \symlist % this page is optional

% \preface % this section is optional
