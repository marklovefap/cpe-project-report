\chapter{\ifenglish Introduction\else บทนำ\fi}

\section{\ifenglish Project rationale\else ที่มาของโครงงาน\fi}
การที่จะสําเร็จการศึกษาได้นั้น นักศึกษาจําเป็นที่จะต้องเรียนวิชาให้ครบเกณฑ์ของแต่ละภาควิชาตามที่สถานศึกษากําหนด
ซึ่งสามารถลงเรียนตามแผนการเรียนรายเทอมตามที่ภาควิชาออกแบบมาให้ได้ แต่การที่นักศึกษาจะลงทะเบียนเรียนในแต่ละเทอมให้
เป็นไปตามแผนการเรียนดังกล่าวนั้นเป็นไปด้วยความยากลําบาก เนื่องจากว่ามีปัจจัยหลายด้านที่ทําให้นักศึกษาไม่สามารถเรียนวิชา
ตามแผนที่กําหนดหรือไม่สามารถสําเร็จกระบวนวิชาได้ในเทอมนั้น เช่น ได้รับอักษร F หรือ W ส่งผลให้วิชาที่เป็นวิชาต่อต้องลงเรียน
ถัดไปในอีกหนึ่งเทอมหรือมากกว่า ซึ่งจะทําให้นักศึกษาเกิดความสับสนในการวางแผนการเรียนของตนเองและไม่มั่นใจว่าตนเอง
สําเร็จกระบวนวิชาในแต่ละหมวดครบแล้วหรือยัง และอาจเกิดปัญหาการลงเรียนวิชาในหมวดนั้นจนเกินความจําเป็น ทําให้เสียเวลา
และหน่วยกิตในเทอมนั้นโดยใช่เหตุ ซึ่งหากจะทําการตรวจสอบด้วยตัวเองโดยการไล่ดูในหมวดนั้นๆว่าลงเรียนวิชาตามเกณฑ์ในหมวด
ครบแล้วหรือยัง จึงเป็นเรื่องที่ยุ่งยาก และเสียเวลาเพราะจําเป็นต้องทําแบบดั่งกล่าวในทุกเทอม
ดังนั้นพวกเราจึงได้คิดหาวิธีแก้ปัญหาโดยการทํา web application ที่ช่วยตรวจสอบกระบวนวิชาในแต่ละหมวดว่า
นักศึกษาสําเร็จวิชาตามเกณฑ์ที่กําหนดในแต่ละหมวดแล้วหรือไม่ เพื่อลดปัญหาการลงเรียนวิชาในหมวดจนเกินความจําเป็น และ
ประหยัดเวลาในการตรวจสอบวิชาที่ตัดสินใจจะลงทะเบียนเรียนของนักศึกษาในแต่ละเทอม นอกจากนี้นักศึกษาสามารถตรวจสอบ
ความคืบหน้าในการสําเร็จการศึกษาตามความถนัดพิเศษ และชนิดของแผนการเรียนที่ตนเองสนใจได้

\section{\ifenglish Objectives\else วัตถุประสงค์ของโครงงาน\fi}
\begin{enumerate}
    \item เพื่อพัฒนา web application ระบบแสดงความคืบหน้าในการสำเร็จการศึกษา สําหรับนักศึกษา อาจารย์ที่ปรึกษา 
    และผู้ดูแลหลักสูตรของมหาวิทยาลัยเชียงใหม่ โดยมีความสามารถดังนี้
\begin{itemize}
    \item ช่วยเเสดงความคืบหน้าในการศึกษาของนักศึกษาได้อย่างถูกต้อง เพื่อเป็นตัวช่วยในการตรวจสอบ เเละลงทะเบียนรายวิชาให้ครบตามกําหนดของหลักสูตร
    \item ช่วยเเบ่งเบาภาระของอาจารย์ที่ปรึกษา ในการตรวจสอบสถานะการสําเร็จการศึกษาของนักศึกษา
    \item ช่วยให้ ผู้พัฒนาระบบสามารถจัดการกับหลักสูตรได้อย่างมีประสิทธิภาพ รองรับหลักสูตรที่มีอยู่ในปัจจุบัน เเละอนาคต
\end{itemize}  
\end{enumerate}

\section{\ifenglish Project scope\else ขอบเขตของโครงงาน\fi}

\subsection{\ifenglish Hardware scope\else ขอบเขตด้านข้อมูล\fi}
\begin{enumerate}
    \item หลักสูตรการศึกษาในระดับปริญญาตรีทุกหลักสูตรของมหาวิทยาลัยเชียงใหม่ ยกเว้นระดับยากมาก ตั้งแต่ปี พ.ศ.2558 เป็นต้นมา
    \item ข้อมูลพื้นฐานทั่วไปของนักศึกษา(OAuth) ข้อมูลพื้นฐานเพิ่มเติมของนักศึกษา(Database ของมหาวิทยาลัย) ข้อมูลการลงทะเบียนรายวิชาของนักศึกษา(Database ของมหาวิทยาลัย)
\end{enumerate}

\subsection{\ifenglish Software scope\else ขอบเขตด้านซอฟต์แวร์\fi}
\begin{itemize}
    \item นักศึกษา
    \begin{enumerate}
        \item ระบบทําหน้าที่หลักในการรายงานความคืบหน้าของการศึกษา จนถึงตรวงสอบการสําเร็จการศึกษา 
        ไม่สามารถถูกใช้ในการวางเเผนการเรียนเเบบรายเทอม ดูตัวต่อวิชา หรือคํานวณเวลาที่จะจบการศึกษาในอนาคตได้
    \end{enumerate}
    \item อาจารย์ที่ปรึกษา
    \begin{enumerate}
        \item อาจารย์ที่ปรึกษาสามารถตรวจสอบได้ว่านักศึกษาที่ตนดูแลอยู่นั้นสําเร็จการศึกษาแล้ว ก็ต่อเมื่อนักศึกษาเข้ามาใช้งานระบบในส่วนของ นักศึกษาก่อนเเล้ว
    \end{enumerate}
    \item ผู้ดูเเลโครงสร้างหลักสูตร
    \begin{enumerate}
        \item ระบบไม่สามารถรองรับในหลักสูตรระดับยากมาก เพียงเเต่เเนวคิดในการออกเเบบระบบ สามารถนําไปใช้ประโยชน์เพิ่มเติมในหลักสูตรระดับยากได้
    \end{enumerate}
\end{itemize}


\section{\ifenglish Expected outcomes\else ประโยชน์ที่ได้รับ\fi}
\begin{itemize}
    \item นักศึกษา
    \begin{enumerate}
        \item สร้างความสะดวกในการตรวจสอบการสําเร็จการศึกษาของตนเอง
        \item ช่วยลดความสับสนในการทําความเข้าใจหลักสูตรการศึกษาที่ตนศึกษาอยู่เพื่อเลือกลงวิชาให้ครบในหมวดที่กําหนด
        \item สามารถเลือกวางแผนการศึกษาในหลักสูตรการศึกษาตามแบบที่ตนเองสนใจได้
        \item สร้างความมั่นใจได้ว่ากระบวนวิชาที่ตนศึกษาอยู่นั้นเป็นไปตามแผนการศึกษาของหลักสูตรหรือไม่
    \end{enumerate}
    \item อาจารย์ที่ปรึกษา
    \begin{enumerate}
        \item สร้างความสะดวกในการตรวจสอบความคืบหน้าการสําเร็จการศึกษาของนักศึกษาทั้งแบบรายบุคคล และแบบ
        กลุ่ม
        
    \end{enumerate}
    \item ผู้ดูเเลโครงสร้างหลักสูตร
    \begin{enumerate}
        \item ช่วยลดเวลาในการจัดการหลักสูตร เนื่องจากเเนวคิดการออกเเบบ Data model 
        ที่จะรวมข้อมูลที่มีการ share เข้าด้วยกัน ทําให้สามารถจัดการข้อมูลเเค่เพียงที่้เดียว เเต่ส่งผลไปหลายหลักสูตร
    \end{enumerate}
\end{itemize}

\section{\ifenglish Technology and tools\else เทคโนโลยีและเครื่องมือที่ใช้\fi}

% \subsection{\ifenglish Hardware technology\else เทคโนโลยีด้านฮาร์ดแวร์\fi}

\subsection{\ifenglish Software technology\else เทคโนโลยีด้านซอฟต์แวร์\fi}
\begin{enumerate}
\item React ใช้เป็นเครื่องมือพัฒนาในส่วนของ frontend เพื่อพัฒนาหน้าเว็บแอพลิเคชันสําหรับแสดงผลข้อมูลต่างๆในส่วนติดต่อกับ ผู้ใช้งาน
\item NodeJS ใช้เป็น backend เพื่อประมวลผล เเละคํานวณข้อมูลเป็นหลัก เป็นศูนย์กลางในการติอต่อ database 
เชื่อมต่อ API ทั้ง REST API เเละ GraphQL API เเละเป็นที่รองรับการติดตั้ง packages เสริมเพิ่มเติมที่ระบบต้องการ
\item ExpressJS ใช้สร้าง server หลักของระบบที่จะตอบรับ request, response ที่เข้ามาจาก client(frontend) 
authentication(OAuth\cite{o}) เเละยังทํางานร่วมกับ REST API\cite{rest} เเละ GraphQL \cite{graphql}
\item REST API ถูกใช้เป็นหลักในการ ติดต่อกับ CMU-OAuth Server เพื่อทําการยืนยันตัวตนเข้าใช้งาน application 
เเละดึงข้อมูลนักศึกษาเบื้องต้นเพื่อมาประมวลผล
\item GraphQL ใช้เป็น API หลักในจัดการข้อมูลของหลักสูตร
\item MongoDB\cite{mongo} ใช้เป็นฐานข้อมูลหลักของระบบ
\end{enumerate}

\section{\ifenglish Project plan\else แผนการดำเนินงาน\fi}

\begin{plan}{6}{2021}{2}{2022}
    \planitem{6}{2021}{6}{2021}{ศึกษาปัญหาจากนักศึกษาและอาจารย์ที่ปรึกษา}
    \planitem{6}{2021}{8}{2021}{ศึกษาโครงสร้างหลักสูตรแต่ละหลักสูตร}
    \planitem{7}{2021}{8}{2021}{ศีกษาเทคโนโลยที่ใช้}
    \planitem{9}{2021}{10}{2021}{เขียนโครงร่างรายงาน}
    \planitem{7}{2021}{9}{2021}{ออกแบบระบบโดยรวม และขั้นตอนการใช้งานระบบ}
    \planitem{9}{2021}{11}{2021}{ออกแบบ UX/UI}
    \planitem{9}{2021}{11}{2021}{ออกแบบ database และ data model}
    \planitem{11}{2021}{2}{2022}{พัฒนาส่วนของ front-end ครั้งที่ 1 (แสดงรายวิชาในหน้านักศึกษา)}
    \planitem{11}{2021}{2}{2022}{พัฒนาส่วนของ back-end ครั้งที่ 1 (สร้าง data model)}
    \planitem{12}{2021}{3}{2022}{พัฒนาส่วนของ front-end ครั้งที่ 2 (แสดงแผนผังต้นไม้สำหรับหนัานักศึกษา)}
    \planitem{12}{2021}{2}{2022}{เรียนรู้การใช้งาน MongoDB และ Mongoose}
\end{plan}
\begin{plan}{3}{2022}{11}{2022}
    \planitem{3}{2022}{5}{2022}{พัฒนาส่วนของ front-end ครั้งที่ 3 (หน้าผู้ดูแลหลักสูตร)}
    \planitem{3}{2022}{5}{2022}{ศึกษา Graphql และ TypeScript สำหรับเขียน back-end}
    \planitem{5}{2022}{5}{2022}{พัฒนาส่วนของ front-end ครั้งที่ 4 (แบบฟอร์มสำหรับเพิ่มคณะและสาขาวิชา)}
    \planitem{6}{2022}{7}{2022}{พัฒนา front-end ครั้งที่ 5 (แบบฟอร์มเลือกแผนการเรียนในหน้านักศึกษา)}
    \planitem{6}{2022}{7}{2022}{พัฒนา back-end (การใช้ filter ในการเลือก path )}
    \planitem{6}{2022}{7}{2022}{เชื่อม front-end เข้ากับ back-end เพื่อทดสอบ flow ของระบบ}
    \planitem{8}{2022}{8}{2022}{เชื่อมต่อข้อมูลกับมหาวิทยาลัยผ่านระบบ OAuth}
    \planitem{8}{2022}{9}{2022}{พัฒนา front-end ครั้งที่ 6 (หน้าสำหรับ login ผ่าน OAuth และหน้าอาจารย์ที่ปรึกษา)}
    \planitem{10}{2022}{10}{2022}{พัฒนา front-end ครั้งที่ 7 (เพิ่มเติมส่วนต่างๆ จากคำแนะนำของอาจารย์ที่ปรึกษาและคณะกรรมการ)}
    \planitem{10}{2022}{10}{2022}{พัฒนา back-end (การตรวจสอบข้อมูลและการสำเร็จการศึกษาของนักศึกษา)}
    \planitem{9}{2022}{11}{2022}{เขียน final report }
    \planitem{11}{2022}{11}{2022}{เก็บ feedback เพิ่มเติมจากการทดลองใช้เว็บแอบพลิเคชัน}
\end{plan}

\section{\ifenglish Roles and responsibilities\else บทบาทและความรับผิดชอบ\fi}
การพัฒนาระบบถูกเเบ่งออกเป็น 2 ส่วน 
\begin{enumerate}
    \item นายชุติพนธ์ วิมลกาญจนา รับหน้าที่ Backend developer รับผิดชอบเรื่อง Database design, API design, Data manipulation
    \item นายอานนท์ รอดตัว รับหน้าที่ Frontend developer รับผิดชอบเรื่อง UX/UI design, client-side web pages, fetch data API
\end{enumerate}

\section{\ifenglish%
Impacts of this project on society, health, safety, legal, and cultural issues
\else%
ผลกระทบด้านสังคม สุขภาพ ความปลอดภัย กฎหมาย และวัฒนธรรม
\fi}
\begin{itemize}
\item ด้านสุขภาพจิตที่ดีขึ้น
\begin{enumerate}
    \item นักศึกษา --
    จากประสบการณ์ของผู้พัฒนา แลผลการสอบถามอย่างโดยละเอียดจากนักศึกษา พบว่า
    หลายๆคนต่างบอกว่ามีความสับสนในแต่ละหมวดวิชา ทําความเข้าใจได้ยาก จึงทําให้หลายคนไม่ค่อย
    สนใจและปล่อยผ่านการทําความเข้าใจหลักสูตรของตนเองทิ้งไป ซึ่งในส่วนนี้จะเกิดผลเสียต่อตัวนักศึกษาเองเนื่องจากจะมีข้อบังคับหรือข้อกําหนดของกระบวนวิชาบางรายวิชาในหมวดต่างที่ต้องลงให้
    ครบ หรือเลือกลงอย่างใดอย่างหนึ่ง ด้วยข้อกําหนดนี้เองอาจทําให้นักศึกษาลงทะเบียนเรียนผิดพลาด
    โดยไม่รู้ตัว และส่งผลให้อาจเรียนไม่จบตามปีการศึกษาที่ตนคาดการไว้ เพราะฉะนั้น web application
    นี้จะช่วยให้นักศึกษามีความเข้าใจในหลักสูตรของตนเอง และเลือกแผนการเรียนที่สนใจได้อย่างถูกต้อง
    รวมไปถึงเกิดความมั่นใจได้ว่าตนเองจะจบการศึกษาได้อย่างแน่นอนตามที่ web application ได้แสดง
    ผลให้เห็น
    
    \item อาจารย์ที่ปรึกษา --
    เนื่องจากว่าอาจารย์ที่ปรึกษาบางท่านเป็นอาจารย์ที่ปรึกษาของนักศึกษามากกว่า 1 หลักสูตร ( เป็น
    อาจารย์ที่ปรึกษาทั้งหลักสูตรปีการศึกษา 2558 และ หลักสูตรปีการศึกษา 2563 ) ซึ่งหลักสูตรแต่ละ
    หลักสูตรจะมีความแต่ต่างกันออกไปมากน้อย จึงอาจทําให้อาจารย์ที่ปรึกษาเกิดความสับสนได้ว่านักศึกษาอยู่คนนั้นอยู่หลักสูตรใด ดังนั้น web application นี้จึงเข้ามาช่วยให้อาจารย์ที่ปรึกษาสามารถ
    ตรวจสอบนักศึกษาคนใดๆที่ตนเองดูแลได้ง่ายขึ้นและรวดเร็วยิ่งขึ้น

    \item ผู้ดูเเลโครงสร้างหลักสูตร --
    จากประสบการณ์ของผู้พัฒนาที่ได้ทําการศึกษาโครงสร้างหลักสูตรของแต่ละคณะ ทุกสาขาวิชา และ
    ในทุกปี การศึกษา พบว่าแต่ละหลักสูตรนั้นมีความซับซ้อนของกระบวนวิชาและโครงสร้างหลักสูตรที่
    แตกต่างกันอย่างมาก และในแต่ละหลักสูตรนั้นจะมีลักษณะเฉพาะของตัวหลักสูตรเองเช่นกัน จึงอาจ
    ทําให้ผู้ดูแลนั้นเกิดความเครียดและสับสนเป็นอย่างมากหากต้องทําการแก้ไข เพิ่ม หรือลบกระบวน
    วิชาภายในหลักสูตร และบางครั้งอาจต้องทําแบบเดิมในหลักสูตรอื่นๆด้วยเป็นจํานวนมาก มิหนําซํ้า
    การจัดเก็บข้อมูลยังทําได้ยากเนื่องจากว่าต้องเก็บข้อมูลโครงสร้างหลักสูตรลงในกระดาษหรืออาจใช้
    ดปรแกรมเสริมอย่างเช่น Google sheet ดังนั้นภาษาพูดหาก web application นี้จะเข้ามาช่วยให้
    ผู้ดูแลเกิดความสะดวกสบายในการแก้ไข เพิ่ม และลบกระบวนวิชาในหลักสูตรได้ง่ายและกระทําการ
    เพียงครั้งเดียวหากเป็นการทํางานเหมือนเดิมซํ้าๆ รวมถึงช่วยในการจัดเก็บและค้นหาหลักสูตรแต่ละ
    อันได้ง่ายและรวดเร็วกว่ารูปแบบเดิม
    
\end{enumerate}    
\end{itemize}
